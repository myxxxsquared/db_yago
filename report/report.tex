% !TEX program = xelatex

\documentclass[12pt]{ctexart}

    \usepackage[colorlinks,linkcolor=red]{hyperref}
    \usepackage{array}
    \usepackage{makecell}
    
    \title{NoSQL数据库存取yago数据\\{\Large 大数据管理技术第二次实习作业}}
    \author{张文杰 1500011394}

    
\begin{document}

\maketitle

\section{介绍}
本次上机作业使用Redis、Mongodb、Cassandra三种数据库,利用上一次实验装好的数据库和数据库驱动程序,在我自己的电脑上,完成了对于yago数据的储存和查询操作。使用的操作系统为Ubuntu 18.04,8G内存,编程语言为Python 3.5。报告中仅展示了部分源代码,详细源代码见Github~\url{https://github.com/myxxxsquared/db_yago}。

\section{实验内容}

实验的内容分为两部分,第一部分为将数据导入到相应的数据库中,第二部分为完成相应的查询,这里规定了四个查询操作。
\begin{enumerate}
    \itemsep0em
    \item[Exp01] 查询S为`Mel\_Thompson'的所有的P和O。
    \item[Exp02] 查询O为`England'的所有的S和P。
    \item[Exp03] 查询含有`graduatedFrom'和`isLeaderOf'的S。
    \item[Exp04] 查询含有`England'最多的S。
\end{enumerate}

\subsection{Redis}

\subsubsection{储存模式}

Redis数据库不能建立除KEY之外的额外索引信息,为了完成指定的查询任务,需要对于S,O,V都有查询,因此需要在Redis数据库中使用冗余的数据,才能完成指定的操作。在这里,我使用了如下的数据存储模式。
\begin{enumerate}
    \itemsep0em
    \item 建立一个名为`~~'(空格)的SET,用于保存所有的S的列表。
    \item 建立名为`S名称~~P名称'(S名称+空格+P名称)的SET,用于保存S在P下的所有O。
    \item 建立名为`O名称'的SET,用于保存所有含有O的S和P。
\end{enumerate}
以如下的数据为例
\begin{verbatim}
Richard_Stallman isLeaderOf Free_Software_Foundation
Richard_Stallman isLeaderOf Cambridge_Bay
Andranik isLeaderOf Armenian_fedayi
\end{verbatim}
其储存在Redis数据库中格式为
\begin{verbatim}
' ':
    ['Richard_Stallman', 'Andranik']
'Richard_Stallman isLeaderOf': 
    ['Free_Software_Foundation', 'Cambridge_Bay']
'Andranik isLeaderOf':
    ['Armenian_fedayi']
'Free_Software_Foundation':
    ['Richard_Stallman isLeaderOf']
'Cambridge_Bay':
    ['Richard_Stallman isLeaderOf']
'Armenian_fedayi':
    ['Andranik']
\end{verbatim}

\subsubsection{查询方法}

对于Redis的查询只能使用KEY来进行,通过si查找P和O,只需要找到所有的形式为`si~~*'的KEY,即使用命令`KEY ``si~~*''',然后使用SMEMBERS命令查询KEY的值即可。

通过oi来查找S和P,只需使用数据库中已有的数据,利用SMEMBERS命令直接查询即可。

通过p1、p2查找所有同时拥有p1、p2的S,可以通过查询所有拥有p1的S和所有拥有p2的S,再求交集即可。查询拥有p1的S可以通过命令`KEY ``*~~p1'''来完成,同理可以查询到所有拥有p2的S,之后通过intersection求交集,可以得到同时拥有两者的S。

通过oi查询拥有这样oi最多的S,可以直接使用SMEMBERS命令查询到所有拥有这样oi的si和pi,经过计数即可找到最多的一项。

\subsubsection{遇到的问题}

我测试使用的电脑内存只有8G,由于储存的数据量较大,而Redis的所有数据均储存在内存中,占用大量的内存。

Redis的数据储存到硬盘的过程直接使用了fork系统调用,在占用大量内存时就会导致内存的overcommit,fork操作失败,失败提示为
\begin{verbatim}
Can't save in background: fork: Cannot allocate memory
\end{verbatim}

这里采用的解决方法是,始终允许系统内核overcommit内存,即写入一个`1'到`/proc/sys/vm/overcommit\_memory'文件内,这样fork系统调用就不会失败。

\subsection{MongoBD}

\subsubsection{储存模式}

在MongoDB中,为了实现以上四种查询,在这里我使用了一种简单类似二维表的存储模式,对于每一条记录,将其储存为一个文档,例如
\begin{verbatim}
Richard_Stallman isLeaderOf Free_Software_Foundation
\end{verbatim}
被储存为
\begin{verbatim}
{
    s: `Richard_Stallman',
    v: `isLeaderOf',
    o: `Free_Software_Foundation'
}
\end{verbatim}
在MongoDB中可以使用索引来加速查询过程,在本次实验中,分别测试了没有索引和有索引的查询效率。

\subsubsection{查询方法}

已知si或oi查询PO和SP,只需使用MongoDB的查询文档
\begin{verbatim}{ s: `si' }\end{verbatim}
或
\begin{verbatim}{ o: `oi' }\end{verbatim}
即可完成查询

通过p1、p2查找所有同时拥有p1、p2的S,类似Redis,可以直接查询出所有含有p1的S和所有p2的S,再取交集。通过oi查询拥有这样oi最多的S,可以通过找出所有的再进行计数。

\subsection{Cassandra}

\subsubsection{储存模式}

这里采用的也是类似二维表的储存方法,建立一个表。由于Cassandra必须有一个无重复的主键,因此多引入一项用于存储编号作为主键。
\begin{verbatim}
create table if not exists yago (
    i int primary key,
    s varchar,
    v varchar,
    o varchar
);
\end{verbatim}
由于Cassandra的应用场景的要求,Cassandra对数据的查询有很多timeout的情形,因此如果不对于数据库建立索引,会导致查询操作超时报错,因此为了满足以上的查询需求,本次实验中仅进行了建立索引后的查询实验,建立索引的操作为
\begin{verbatim}
create index if not exists sindex on yago (s);
create index if not exists vindex on yago (v);
create index if not exists oindex on yago (o);
\end{verbatim}

\subsubsection{查询方法}

由于数据存储模式与Cassandra基本相同,查询方法也基本相同,只不过是把查询文档改为CQL语句。
\begin{verbatim}
select s, v, o from yago where s = %s;
select s, v, o from yago where v = %s;
select s, v, o from yago where o = %s;
\end{verbatim}

\section{实验结果和分析}

四种操作分别在三个不同的数据库上进行操作,得到的操作时间如表\ref{tab}所示。

比较MongoDB的有无索引,有索引的查询速度总是大大快于无索引的查询速度。

Exp02和Exp04中涉及了大量的数据查询,这是由于以`England'为O的记录很多,在这样的实验中,Redis表现出较高的效率,说明Redis适用于大量简单的查询操作。在这两个测试上,MongoDB的效率略低于Redis,但是也远远高于Cassandra。

在Exp01和Exp03,MongoDB的速度快于Redis,这里Redis需要进行KEY查询,一定程度上影响了速度,使得MongoDB速度相对较快。

Cassandra数据库的表现除了数据量较小的Exp01外,表现普遍较差,可能是由于Cassandra数据库的设计中有冗余储存等技术,Cassandra查询数据速度慢与另外两个数据库。

\begin{table}
    \centering
    \caption{查询操作时间(单位 秒)\label{tab}}
	\begin{tabular}{|c|p{2.5cm}<{\centering}|p{2.5cm}<{\centering}|p{2.5cm}<{\centering}|p{2.5cm}<{\centering}|}
		\hline
		      & Redis & \makecell{MongoDB\\(有索引)} & \makecell{MongoDB\\(无索引)} & Cassandra \\
		\hline
		Exp01 & 1.351 & \textbf{0.022} & 10.41 & 0.040 \\
		Exp02 & \textbf{0.779} & 1.210 & 5.799 & 30.79 \\
		Exp03 & 8.378 & \textbf{0.982} & 10.25 & 14.63 \\
		Exp04 & \textbf{1.024} & 1.046 & 6.110 & 24.28 \\
		\hline
	\end{tabular}
\end{table}

\section{个人体会}

在本次的上机实验中,三个数据库各有各的特点。Redis全部使用内存存储,占用大量内存空间。MongoDB支持一定的数据结构,使得操作和查询变得更加容易。Cassandra更适用于分布式存储,对于本实验中单一结点的存储体验较差。

个人感觉而言,还是更喜欢MongoDB,因为MongoDB操作简单,支持比较灵活的数据结构操作和查询,并且不容易发生问题。

\end{document}